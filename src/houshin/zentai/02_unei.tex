\subsection*{運営方針}

%\writtenBy{\president}{大野}{直哉}
\writtenBy{\subPresident}{大野}{直哉}
%\writtenBy{\firstGrade}{大野}{直哉}
%\writtenBy{\secondGrade}{大野}{直哉}
%\writtenBy{\thirdGrade}{大野}{直哉}
%\writtenBy{\fourthGrade}{大野}{直哉}

2023年度秋学期の運営に関して以下の5点から方針を述べる.
\begin{itemize}
    \item 定例会議
    \item 上回生会議
    \item 局
    \item 企画
    \item 運営サポート
\end{itemize}

\subsubsection*{定例会議}
春学期と同様に毎週木曜日に対面で開催する.
出席が不可能な会員が多いなどの問題が発生した場合は,開催日時の変更を行う.
必要に応じてSlackやDiscordで資料や議題の周知と共有を行う.

\subsubsection*{上回生会議}
春学期同様に必要に応じて随時開催する.
各局の局長またはその代理人が必ず参加するものとする.
各局長は局員が上回生会議における議題の内容を把握できるようにする.
議決権のない会員に対しても上回生会議に参加可能である旨を告知する.

\subsubsection*{局}
局配属に関しては遅れが生じているため,早急に取り組む.
上回生会議にて都度局での状況を確認し,必要に応じて局会議を開催する.

\subsubsection*{企画}
担当者は,会員2名以上であることが望ましい.
必要に応じて執行部や2022年度担当者がフォローする.
企画の進捗は上回生会議で確認する.
KPTに関しては,基本的に上回生会議で行うが,主催が局である場合は局会議で行う.

\subsubsection*{運営サポート}
春学期同様,\thirdGrade{}がサポートするが,春学期の状況などを顧み,より\secondGrade{}中心の運営を行っていく.
秋学期からは\thirdGrade{}が研究室に配属されるため,必要な時は可能な限り早めの連絡を心がける.
