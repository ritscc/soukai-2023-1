\subsection*{プロジェクト活動方針}

%\writtenBy{\president}{尾﨑}{真央}
%\writtenBy{\subPresident}{尾﨑}{真央}
%\writtenBy{\firstGrade}{尾﨑}{真央}
\writtenBy{\secondGrade}{尾崎}{真央}
%\writtenBy{\thirdGrade}{尾﨑}{真央}
%\writtenBy{\fourthGrade}{尾﨑}{真央}

目的
    プロジェクト活動の目的は,情報科学の研究をし,その成果の発表を活動の基本として会員相互の親睦を図るとともに学術文化の創造と発展に寄与することとする
目標
    個人のみならずグループ活動としての経験を得る
    活動を通して技術力の向上を図る
    活動によって得られた成果を本会 Web サイトを通して公開する
プロジェクトの内容
    プロジェクト内容は基本的に学習または,研究要素を含むものとする
プロジェクトの設立
    通年はそのまま
    新規プロジェクトの設立を認めます
    2023年秋学期入会者への対応
    プロジェクトは以下の条件を満たした場合に設立できるものとする.
    リーダーの作成した企画書が,上回生会議で承認されること
    メンバーが,募集終了時点でリーダーを含め 3 人以上であること
    リーダーが,入会して半年以上経過していること
    1 人の会員が,複数のプロジェクトのリーダを担当していないこと
    1 人が一度に複数企画書を提出していないこと
メンバー募集
    定例会議でリーダーがそれぞれプロジェクトの説明を行い,その次の定例会議までに\fourthGrade{}を除く全会員が一つ以上のプロジェクトに所属する.

週報
    春学期と同様にする
    活動後の週報の提出を義務とする
    上回生会議にて週報の確認を確実に実施する
    
    
プロジェクト解散
    プロジェクトに配属後,班員が3人未満もしくはリーダーが欠けた場合,そのプロジェクトは趣意書をもって理由を記述したのちに,上回生会議によって解散される
    
プロジェクト発表会
    プロジェクトで得た知見や技術を共有する場として成果発表を行う
    発表形式
    事前に配布された報告書をその場で読む時間を設ける
    スライド発表,質疑応答
    対面orオンラインは状況による
    できるなら追い込み合宿をやる
    10月中に外出を伴う課外自主活動自粛要請が無効になった場合は実施する方向で
    対面の場合,報告書を2部印刷する
    報告書のレビューはPDFで行う
    報告書は予めテンプレートを作成しておき,プロジェクトリーダーに配布する
    配布時期は,定例会議にて配布,そして連絡
        
Web サイトへの公開
    プロジェクト活動で作成した報告書を本会 Web サイトで公開
    報告形式などを研究推進局が決定
    最終的にはPDFで提出してもらう
    それまでの執筆形式は自由
    報告書をプロジェクト発表会中にレビューして修正
    その際,報告書の修正期限を設ける
    修正された報告書を渉外局に依頼し,本会Web公開する
    報告書以外の制作物
    プロジェクトの報告書では載せられない制作物はWebサイトでの公開を推奨
    義務付けつるとプロジェクト設立数が減るかもしれないから
    制作物の,著作権などのチェックを上回生会議で行う
        
プロジェクトの運営
    週報が出ていないまたは週報に活動の継続が難しい旨が記述されていた場合,プロジェクトのリーダーを上回生会議に招集し,プロジェクトの存続を問う
    その際,リーダーが上回生会議に来ないまたはプロジェクト存続の意思がないならば,全ての班員を上回生会議に招集し,リーダーを受け継ぐ意志があるものが存在し,かつ班員としてプロジェクト活動を行う意志のあるものがリーダーを含め 3 人以上存在する場合に限り,プロジェクトを存続するものとする.
    この時点でプロジェクトの存続が決定しなかった場合,プロジェクトは解散となる
    秋学期の最後になった場合は?
        理想は残りの班員で成果報告書を完成
        上回生会議で柔軟に対応
        
    
