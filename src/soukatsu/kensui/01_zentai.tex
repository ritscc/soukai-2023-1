\subsection*{全体総括}

\writtenBy{\kensuiChief}{梶原}{悠人}
%\writtenBy{\kensuiStaff}{梶原}{悠人}

\begin{itemize}
    \item 平常活動の支援
    \item 会員が興味関心のある活動ができる環境づくり
    \item 発信力を養うための環境づくり
  \end{itemize}
  
  \subsubsection*{平常活動の支援}
  平常活動の支援に関しては,週報を用いてプロジェクト活動の進捗確認や問題の有無の確認を行う.
  
  プロジェクト活動の進捗管理は,週報を用いてプロジェクト活動の進捗確認や問題の有無の確認を行い,
  問題が確認された場合は,それを上回生会議の議題に上げることで問題の解決を図るという方針であった.
  これに関して,報が記述されることが無かったため,プロジェクト活動の進捗管理を行えなかった.
  
  また,追い込み合宿やプロジェクト発表会は行われていない.
  
  \subsubsection*{会員が興味関心のある活動ができる環境づくり}
  会員が興味関心のある活動ができる環境づくりに関しては,プロジェクト活動を行った.
  
  \subsubsection*{発信力を養うための環境づくり}
  発信力を養うための環境づくりに関しては,LTの運営を行った.
  
  LTは毎週の定例会議中で行われた.
  
  2023年度春学期は,春学期後半になると,LTを遅延して行われたものや,最後までLTを行えなかったものが多かった.
  
  また,例年通りLTは録画を撮りRCC NASに載せている.
  
