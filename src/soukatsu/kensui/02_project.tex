\subsection*{プロジェクト総括}

\writtenBy{\kensuiChief}{梶原}{悠人}
%\writtenBy{\kensuiStaff}{梶原}{悠人}

本項では本局におけるプロジェクト活動業務に関する2022年度秋学期の総括を以下の点において述べる.

\begin{itemize}
  \item 企画書の募集
  \item 週報の回収・催促
  \item 会員のプロジェクト管理
  \item 発表の機会の提供
  \item 報告書の管理
\end{itemize}

\subsubsection*{企画書の募集}

2023年度春学期に設立したプロジェクトは四つである.
企画書を局会議と上回生会議で確認を行い,全ての企画書に問題が無かったため,全てのプロジェクトを設立した.

\subsubsection*{週報の回収・催促}

各プロジェクトリーダーは,プロジェクト活動の進捗確認や問題の有無の確認を行うために,
週報の提出が義務付けられている.
週報の回収にはGoogleフォームが用いられるが,活用されることはなかった.

\subsubsection*{会員のプロジェクト管理}

本局は,各会員がどのプロジェクトに所属しているかを把握し,
プロジェクトが途中で終了した場合などに所属していた会員のプロジェクト異動などを管理している.
画像生成班はPCの組み立てで苦戦していたため,プロジェクト活動が行われることがなかった.
VR班はプロジェクトリーダーが多忙のため,活動を行うことが困難であった.
\subsubsection*{発表の機会の提供}

プロジェクト活動の成果発表を例年プロジェクト発表会を通じて行っているが,2023年度は行われなかった.

\subsubsection*{報告書の管理}

プロジェクト活動の成果を記録として残すため,各プロジェクトリーダーには報告書の提出を義務としている.
しかし,2023年度はリマインドなどを行っていなかったため,提出されることはなかった.
