\subsection*{運営総括}

%\writtenBy{\president}{大野}{直哉}
\writtenBy{\subPresident}{大野}{直哉}
%\writtenBy{\firstGrade}{大野}{直哉}
%\writtenBy{\secondGrade}{大野}{直哉}
%\writtenBy{\thirdGrade}{大野}{直哉}
%\writtenBy{\fourthGrade}{大野}{直哉}

2023年度春学期の運営を以下の4点から述べる.
\begin{itemize}
    \item 定例会議
    \item 上回生会議
    \item 局
    \item 企画
\end{itemize}

\subsubsection*{定例会議}
毎週木曜日に対面で開催した.
2023年度は対面での開催のみで,Zoomなどを利用したオンラインでの配信は行われなかった.
内容は例年通り執行部及び局からの連絡,会員によるLTであった.
定例会議を開催する教室の予約は\kensuiChief{}が行った.

\subsubsection*{上回生会議}
随時議題が上がり次第,Zoomを利用して開催した.
参加に関して,欠席した場合の代理人が立てられることは無かった.
会議は執行部のみで行い,他の会員の参加は特に無かった.
内容に関しては,議題があれば随時共有がされていたため,運営は円滑に行われた.

\subsubsection*{局}
局の運営については局会議と局配属の2点から述べる.
\paragraph*{局会議}
局会議が開催された局は少なかったが,上回生会議で代わりの議論は行われていた.
またSlcakなどで必要な連絡はされていたため,運営に支障はなかった.
\paragraph*{局配属}
局配属は新入生の希望調査が遅れており,局配属は行われなかった.

\subsubsection*{企画}
企画の担当者は基本的に2名設けていた.
運営の主体は\secondGrade{}で行われており,必要に応じて\thirdGrade{}がサポートを行った.
企画後のKPTは企画担当者と執行部で行ったが,一部の企画では行えていなかった.
