\subsection*{立命の家総括}
\subsection*{立命の家総括}

%\writtenBy{\president}{新井}{康平}
%\writtenBy{\subPresident}{新井}{康平}
%\writtenBy{\firstGrade}{新井}{康平}
\writtenBy{\secondGrade}{新井}{康平}
%\writtenBy{\thirdGrade}{新井}{康平}
%\writtenBy{\fourthGrade}{新井}{康平}

\subsection*{概要}
立命の家は毎年立命館大学で開催される小学生を対象とした企画である.
2023年度は8月23日,8月24日の二日間開催された.

\subsection*{目的と目標}
\begin{itemize}
    \item 学術部公認団体として求められる還元活動の義務を果たすこと
    \item 本企画に参加する小学生に本会の活動と情報技術に興味・関心を向けてもらえること
    \item 小学生に教える体験を通して会員の教える能力の向上をはかり今後の還元活動を円滑にできるようになること
\end{itemize}
\subsection*{実施内容と提案}
2023年度は二日間にわたって開催され,両日対面で行ったC
Scratchを用いたプログラミング勉強会を行った.
2022年度は参加者が作りたい機能を完成されたゲームに追加していくことを行った.
しかし.課題が容易であり.時間を持て余す参加者が出てしまったため.2023年度は一からプログラムを作っていくという課題に変更した.

\subsection*{対象}
5年生以上推奨にしたため企画を滞りなく行うことができた.
また2日とも参加人数を7まで減らした.

\subsection*{役割}
担当者2人で役割分担してうまく運営することができた.

\subsection*{改善点}
課題が難しく.最後までプログラムを作り終えることが出来ない参加者が発生してしまった.
企画への参加者が少なく.運営がぎりぎりになってしまった.
