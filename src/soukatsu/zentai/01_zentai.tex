\subsection*{2023年度春学期活動総括}

\writtenBy{\president}{羽田}{秀平}
%\writtenBy{\subPresident}{羽田}{秀平}
%\writtenBy{\firstGrade}{羽田}{秀平}
%\writtenBy{\secondGrade}{羽田}{秀平}
%\writtenBy{\thirdGrade}{羽田}{秀平}
%\writtenBy{\fourthGrade}{羽田}{秀平}

本会の目的である「情報科学の研究,及びその成果の発表を活動の基本に会員相互の親睦を図り,学術文化の創造と発展に寄与する」ことを達成するため,方針として以下の六つを立てた.
これらについてそれぞれ評価を行うことで2023年度春学期の総括とする.

\begin{itemize}
  \item 親睦を深める
  \item 規律ある行動
  \item 自己発信力の向上
  \item 会員間の技術向上
  \item 外部への情報発信
  \item 持続可能な運営
\end{itemize}

\subsubsection*{親睦を深める}
2023年度春学期活動では新歓交流会,プロジェクト活動を実施することによって会員間の親睦を図った.

尚,2022年度春学期活動においては対面活動が許可されているため,活動形式は状況に応じてオンライン,対面,あるいは両方で行う形をとった.

新歓交流会にも多くの新入生が参加し,自己紹介後には外部で食事会を行い交流の場を設けた.
上回生との会話を通して,良い影響があったように思われる.

プロジェクト活動では,各々の班で共同開発や発表が活発に行われた.

\subsubsection*{規律ある行動}
本項では,遅刻・欠席連絡と備品整備,サークルルームの使用方法の三つについて評価する.

遅刻・欠席連絡については,概ねDiscordの専用チャンネルにおいて行われていた.
例年に比べ欠席連絡の頻度が少なかったため,改善の余地がある.
また開始時刻を過ぎてからの連絡は少なかった.

また,サークルルームは\secondGrade{}以上の利用が多く,\firstGrade{}の利用はやや少なかった.

\subsubsection*{自己発信力の向上}
自己発信力の向上の機会として,2023年度春学期活動では,LTを行った.

LTは,割り当てられていた会員の全員が発表したため,定例会議での発表は充実していた.
しかし,LTの遅延など一部の会員は発表を行わなかったがために,発表が行なわれず,上回生の活動報告を行うなど臨時の対応が求められたため,各週ごとにLTを担当している会員に連絡するなど改善の必要がある.

\subsubsection*{会員間の技術向上}
会全体の技術力を向上させることを目的として,LTやプロジェクト活動を開催した.

LTの内容には簡単のものが多く,内容が新入生には伝わりやすかったと考えている.
しかし,会員が自身の興味対象について深い内容を扱ったものが少なかったことも否めない.

プロジェクト活動は対面で行ったが,進捗管理に問題があり,完了することはできなかった.

\subsubsection*{外部への情報発信}
会外へ活動を発信する機会として,主に本会Webサイトと会公式Twitter,KC3が挙げられる.

会公式Twitterでは,LTやイベントが行われる度にその様子が発信された.
こちらは頻度が十分であり,内容も適切であった.

また,KC3では本会の活動紹介や勉強会開催を通して,本会の活動を発信した.
懇親会においては,複数の会員が参加し,自身の興味分野について発信した.

\subsubsection*{持続可能な運営}
始めに,今後の本会の活動拠点について述べる.
情報理工学部の移転に際して本会の活動拠点が変更される可能性があるので今後も学生オフィスとの連絡を行う必要がある.
現時点ではOICとBKCの両方を活動拠点とすることが,最も活動に適していると考えているが,学生オフィスの意見は一つの活動拠点に絞ることが望ましいとのことであった.
一方で,OICの活動拠点について,BKCのものと異なってPJ団体との共同スペースになっているので,学術公認団体に所属する本会は遺憾の意を示している.
これはPJ団体と学術公認団体が立命館大学に貢献している程度を考慮すると当然だと考えられる.
よって,今後もOICとBKCの両方を活動拠点とすることを第一の目標とする.
またこれを達成するために,学生オフィスとの連絡を続けるともに,学術公認団体を管理している機関との連絡を行う必要がある.

次に新入生の歓迎について行なったことを述べる.
ポスターの刑事について.申請方法がWebで行なわれるようになったため,前年のように対応が遅れることなく,掲示することができた.
ブース出展に関しては執行委員長の引継ぎが疎かになったため,春学期の開講期のみ行った.
新入生をより獲得するために,開講期前にも行うべきである.
また大学側の新歓企画には参加しなかったが,前年度の新勧企画では約20名近くの新入生が参加したので,今後は参加することが望ましい.
そのためにも,新執行部への引継ぎを前執行部がより丁寧に行い,新勧期において各々の会員の役割を新執行部が早期に決定することで更なる新入生の獲得が見込める.
また夏季休暇期間にも本会に参加したい新入生がメールで問い合わせをしてきたが,対応が遅れたので渉外局の対応を早める必要がある.
続いて,情報理工学部の新入生について,情報理工学部の移転に伴い,情報理工学部の新入生はPJ団体に興味を示していたが,本会は大変少なく感じた.
これは,新入生オリエンテーションと呼ばれる,全ての新入生が集まって行なわれる説明会があり,そこでPJ団体の紹介が行われたためであると考えられる.
よって本会も新入生オリエンテーションに参加することが今後の運営において極めて重要であると考えており,周到な準備を事前に行うことが今後の運営に大きな影響を与えると考えられる.

最後にイベントについて述べる.
2023年度春学期及び夏季休暇ではOBOG会と立命の家を実施した.
OBOG会については,OBOG会を運営されているOBの方と密に連絡を取り合うことで円滑に進行することができたため,参加者人数が約50人となり,大変盛り上がった.
アンケートの結果を確認すると現役生及びOBの方からも好評であった.
次回の開催については2025年度であることが決定したので,引継ぎ資料をGoogleドライブに残した.
よって十分目標を達成することができた.