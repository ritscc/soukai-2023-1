\subsection*{新歓総括}

%\writtenBy{\president}{冨高}{麟太郎}
%\writtenBy{\subPresident}{冨高}{麟太郎}
%\writtenBy{\firstGrade}{冨高}{麟太郎}
%\writtenBy{\secondGrade}{冨高}{麟太郎}
\writtenBy{\thirdGrade}{冨高}{麟太郎}
%\writtenBy{\fourthGrade}{冨高}{麟太郎}

2023年度春学期の新歓の目的は,以下の2点であった.

\begin{itemize}
    \item 新入生に会の活動内容や活動方針について知ってもらう
    \item 新入生に会に興味を持ってもらう
\end{itemize}

これらの目的を達成するため,以下の4点の目標を掲げた.

\begin{itemize}
    \item LT会に参加してもらう
    \item 気軽にサークルルームに来てもらう
    \item 新入生に本会でやりたい事を見つけてもらう
    \item 新入生の中長期的な定着
\end{itemize}

これらの目標を達成するため,まず大学が主催する対面ブースに参加した.
対面ブースは主催を変えて複数回行われたが,できる限り参加した.
ブースでは活動の紹介や,サークルルームの場所の説明,後日開催の新歓団体企画への誘導を行った.

対面ブースの他に大学主催の新歓団体企画に応募,開催した.
新歓団体企画では活動紹介の他,LT会を開催した.
ブースやTwitterで告知が功を奏し,多くの新入生に参加してもらえた.

二次企画は基本的にサークルルームで行った.
二次企画によって新入生に本会でやりたい事を見つけてもらう他,気軽にサークルルームにきてもらうきっかけにもなったと考えられる.

以上の対面ブース,新歓団体企画,二次企画での活動により,2023年度における新歓の目的は達成されたと考えられる.
